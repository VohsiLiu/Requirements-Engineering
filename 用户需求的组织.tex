\section{基于用例/场景模型展开用户需求获取}

\subsection{用户需求的组织}
\begin{itemize}
    \item 场景/用例模型驱动
    \begin{itemize}
        \item 整理和归类需求获取行为得到的信息(框架)
        \item 指导和组织需求获取行为的开展
        \item 为详细信息的分析提供背景基础和上下文知识 
    \end{itemize}
    \item 承上启下
    \vspace{-0.8em}
	\begin{multicols}{2}
    \begin{itemize}
        \item 展开上一层(业务需求)
        \item 准备下一层的展开(系统级需求)
    \end{itemize}
	\end{multicols}
\end{itemize}

\subsection{场景}
\begin{itemize}
    \item $[$Zorman 1995]将场景定义为对系统和环境行为的局部描述
    \item $[$Plihon 1998]将场景定义为对行为或者事件序列的描述,序列中的行为和事件是系统需要完成的一个任务的特殊示例。
    \item $[$Jarke 1996]认为场景包含有行为序列和行为发生的环境,环境描述了行为的主体、客体和上下文设置
    \item 以上的描述都不足以作为场景的准确定义,人们也很难给场景下一个非常准确的定义[Rolland 1998a]
    场景强调系统同外部环境互动以完成预期任务
\end{itemize}

\subsection{场景与用例在需求工程中的定位}
\begin{itemize}
    \item 场景
    \begin{itemize}
        \item 具有重点描述真实世界的特征,利用情景、行为者之间的交互、事件随时间的演化等方式来叙述性的描述系统的使用
    \end{itemize}
    \item 用例
    \begin{itemize}
        \item 相关场景集合的叙述性的文本描述 
        \item 最早在Objectory方法[Jacobson 1992]中提出的
    \end{itemize}
    \item UML将用例定义为“在系统(或者子系统或者类)和外部对象的交互当中所执行的行为序列的描述,包括各种不同的序列和错误的序列,它们能够联合提供一种有价值的服务”[Rumbaugh 2004]。
    \item 商业模式中的场景一般需要多个用例来支撑(商业模式设计:讲故事$\rightarrow$场景)
\end{itemize}


