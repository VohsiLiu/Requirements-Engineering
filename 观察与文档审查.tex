\section{观察与文档审查}

\subsection{观察的情境适用性}
观察应用于用户无法完成主动的信息告知的情况下,目前常见的观察方法有以下几种
\vspace{-0.8em}
\begin{multicols}{3}
    \begin{itemize}
        \item 采样观察
        \item 民族志
        \item 话语分析
        \item 协议分析
        \item 任务分析
    \end{itemize}
\end{multicols}
\vspace{-1em}

某些事件只有和它们发生时的具体环境联系起来,才能得到理解,情景性的重要性质主要有以下几个
\vspace{-0.8em}
\begin{multicols}{2}
    \begin{itemize}
        \item 突现:集体促成 ,互动中突现 
        \item 局部:特定的上下文环境 
        \item 暂时:演进过程中的一刻
        \item 涉身:参与者的认知和能力受限
        \item 开放:业务不确定并开放,以后完善
        \item 模糊:基于潜在知识,尚未明确表达
    \end{itemize}
\end{multicols}
\vspace{-1em}

观察方法对情景性问题的解决
\begin{table}[H]
    \centering
    \begin{tabular}{|c|c|l|}
    \hline
    方法                    & 情景性性质 & \multicolumn{1}{c|}{描述}          \\ \hline
    \multirow{3}{*}{采样观察} & 局部    & 对工作进行一段时间的观察,发现其中的异常处理           \\ \cline{2-3} 
                          & 暂时    & 对实际工作进行观察,发现并纠正其与规章、手册或者用户意识中的不同 \\ \cline{2-3} 
                          & 模糊    & 观察特殊事件的进行,发现用户工作中的潜在知识           \\ \hline
    \multirow{5}{*}{民族志}  & 突现    & 通过观察,分析群体的互动,理解复杂的协同事件           \\ \cline{2-3} 
                          & 局部    & 长时间的观察,可以发现各种情况下的异常处理和特殊情况       \\ \cline{2-3} 
                          & 暂时    & 对实际工作进行观察,发现并纠正其与规章、手册或者用户意识中的不同 \\ \cline{2-3} 
                          & 涉身    & 在观察中学习,了解用户本身的认知和能力              \\ \cline{2-3} 
                          & 模糊    & 了解各种基础的细节,能够发现用户工作中的潜在知识         \\ \hline
    会话分析                  & 涉身    & 通过分析用户交谈发现用户的认知和能力               \\ \hline
    协议分析                  & 模糊    & 发现用户工作中的潜在知识                     \\ \hline
    \end{tabular}
    \vspace{-1em}
\end{table}

观察方法解决的问题
\begin{itemize}
    \item 理解复杂的协同事件:突现,民族志 
    \item 获取工作中的异常处理:局部,采样观察、民族志
    \item 获取与用户认知不一致的实际知识:暂时,采样观察、民族志 
    \item 了解用户的认知:涉身,民族志、话语分析
    \item 获取默认知识:模糊,采样观察、民族志、协议分析
\end{itemize}


\subsection{观察方法的应用}

\subsubsection{采样观察}
\vspace{-0.5em}
\begin{table}[H]
    \centering
    \begin{tabular}{|c|l|l|}
    \hline
    \multicolumn{1}{|l|}{} & \multicolumn{1}{c|}{时间采样}                                                         & \multicolumn{1}{c|}{事件采样}                                               \\ \hline
    优点                     & \begin{tabular}[c]{@{}l@{}}\ding{172}通过随机的观察减少偏差\\ \ding{173}对频繁发生事件取代表性事件进行观察\end{tabular}         & \begin{tabular}[c]{@{}l@{}}\ding{172}允许在行为展开过程中观察\\ \ding{173}允许对指定的重要事件进行观察\end{tabular} \\ \hline
    缺点                     & \begin{tabular}[c]{@{}l@{}}\ding{172}用分段的方式来收集数据不能提供全面信息的时间\\ \ding{173}漏掉不经常发生却很重要的事件\end{tabular} & \begin{tabular}[c]{@{}l@{}}\ding{172}消耗大量时间\\ \ding{173}漏掉频繁发生事件的代表性样本\end{tabular}       \\ \hline
    \begin{tabular}[c]{@{}c@{}}适用\\ 情景\end{tabular}                   & \begin{tabular}[c]{@{}l@{}}\ding{172}发现异常流程\\ \ding{173}验证用户知识和实际工作的一致性\end{tabular}                & \begin{tabular}[c]{@{}l@{}}\ding{172}获取默认知识\\ \ding{173}验证用户知识和实际工作的一致性\end{tabular}      \\ \hline
    \end{tabular}
\vspace{-1em} 
\end{table}

\subsubsection{民族志}
对一些复杂的协同工作而言,工作的协同安排具有一定的社会性,是按照社会化的方式组织的,也就是说复杂的协同 工作具有突现的情景性。民族志可以帮助开发者了解这些工作的社会性因素,解决突现的情景性因素。

民族志的优点
\vspace{-0.8em}
\begin{multicols}{2}
    \begin{itemize}
        \item 能够得到信息的深度理解
        \item 能够让真实世界的社会性因素可见化
        \item 打破人们已有的一些错误假设和错误观念
    \end{itemize}
\end{multicols}
\vspace{-1em}

民族志的缺点
\vspace{-0.8em}
\begin{multicols}{2}
    \begin{itemize}
        \item 需要耗费很多的时间
        \item 调研结果很难传递到开发过程
    \end{itemize}
\end{multicols}
\vspace{-1em}

在民族志当中,需求工程师和开发者需要关注3方面的内容
\begin{itemize}
    \item 工作的分布式协同(Distributed Coordination)
    \begin{itemize}
        \item 要特别注意那些利用物件实现的协同和创建这些物件的文书工作
    \end{itemize}
    \item 工作的计划和程序(Plans and Procedures)
    \begin{itemize}
        \item 关注它们在组织活动中的应用方式
        \item 发现实际工作和文档化程序之间存在的偏离
    \end{itemize}
    \item  工作的意识(Awareness of Work)
    \begin{itemize}
        \item 活动是如何对协同中的其他人可见或者可理解的?
    \end{itemize}
\end{itemize}

适用普通民族志的规则
\vspace{-0.8em}
\begin{multicols}{2}
    \begin{itemize}
        \item 应该定期的记录发现 
        \item 尽快的记录可能会在观察过程中发生的面谈 
        \item 定期的复查和更新自己的想法 
        \item 确定管理海量数据的应对策略 
    \end{itemize}
\end{multicols}
\vspace{-1em}

\subsection{文档审查方法的应用}
\begin{table}[H]
    \vspace{-0.5em}
    \centering
    \begin{tabular}{|c|c|l|}
    \hline
    文档类型                                                   & 文档审查方法 & \multicolumn{1}{c|}{描述}                                                                                               \\ \hline
    \begin{tabular}[c]{@{}c@{}}相关产品的需求\\ 规格说明\end{tabular} & 需求重用   & \begin{tabular}[c]{@{}l@{}}分析相关产品的规格说明,发现可以移植到到\\ 新产品中的需求信息,进行需求的重用\\ $\bullet$\ 问题域信息\\ $\bullet$\ 用户界面特征\\ $\bullet$\ 业务需求、组织策略、政策法规\end{tabular} \\ \hline
    硬数据                                                    & 文档分析   & \begin{tabular}[c]{@{}l@{}}阅读、研究得到的硬数据,从中发现需求信息\\ $\bullet$\ 问题域信息\\ $\bullet$\ 工作流程\\ $\bullet$\ 业务细节\end{tabular}                                 \\ \hline
    客户的需求文档                                                & 需求剥离   & \begin{tabular}[c]{@{}l@{}}抽取客户的需求文档中的需求描述\\ $\bullet$\ 粗粒度需求\end{tabular}                                                      \\ \hline
    \end{tabular}
    \vspace{-1em}
\end{table}

\vspace{-0.5em}
\begin{shaded}

\subsubsection*{附:硬数据及硬数据采样}

人们在进行实际工作时会产生各种各样的表格和文档资料,这些表格和文档资料往往是用户对实际业务进行加工和抽象之后的结果,是一种精化过的知识。因此,在研究一个现有系统时,有经验的需求工程师总是会从现有文档中获取事实,理解问题域。这些文档资料被称为硬数据。

[Kendall 2002]将常见的硬数据分为定量硬数据和定性硬数据两种类型
\begin{itemize}
    \item 定量硬数据
    \begin{itemize}
        \item 数据收集表格
        \begin{itemize}
            \item 反映了组织的信息流 
            \item 收集正在使用的每张空白表格表格、填写和分发说明 
            \item 对比填写好的表格
            \begin{itemize}
                \item 表格中是否有从来都不填写的数据项
                \item 应该收到表格的人是否真的收到了
                \item 他们是否按照正常程序使用、存储和丢弃表格
            \end{itemize}
        \end{itemize}
        \item 统计报表
        \begin{itemize}
            \item 反映了组织过去的主要业务和业务目标
            \item 统计规则也是一种丰富的知识,统计项分解为细节业务数据的过程往往也就是组织目标分解到具体业务的过程
            \item 根据实际工作填写过的统计报表,就可以发现组织实际的业务执行状况,从中发现组织面临的具体问题  
        \end{itemize}
    \end{itemize}
    \item 定性硬数据
    \begin{itemize}
        \item 整个组织的描述文档 
        \begin{itemize}
            \item 组织结构图:帮助发现项目的关键涉众
            \item 门户网站:反映组织的业务开展状况
        \end{itemize}
        \item 业务指导文档
        \begin{itemize}
            \item  工作指南和规章手册:解释业务的详细执行过程,反映业务的具体细节
        \end{itemize}
        \item 业务备忘
        \begin{itemize}
            \item 反映业务的实际执行情况
            \item 形成对组织工作过程的清晰理解
        \end{itemize}
    \end{itemize}
\end{itemize}

抽样时样本大小的选择取决于需求工程师希望样本具有多大的代表性。用于确定样本大小($SS$)的一个简单而有效的公式是
$$SS = p\times (1-p) \times \mbox{(确定性因子/可接受的错误)}^2$$
其中$p$是差异样本比例,未知的情况下设为$0.25$

\begin{wraptable}{r}{6.5cm}
    \centering
    \vspace{-1.5em}
    \begin{tabular}{|c|c|}
    \hline
    期望的确定性 & 确定性因子 \\ \hline
    95\% & 1.960  \\ \hline
    90\% & 1.645  \\ \hline
    85\% & 1.281  \\ \hline
    \end{tabular}
    \caption*{常见的确定性因子}
    \vspace{-3em}
\end{wraptable}

例:采样数量示例
\begin{itemize}
    \item 每10张发票中就有1张发票与常规情况不同 
    \item 希望发票样本中包含所有的情况具有90\%的确定性 
\end{itemize}
则选取的样本大小为样本大小为$SS=0.10\times (1-0.10)\times(1.645/0.10)^2=25$

如何选择这25张发票呢?两种常用的抽样技术是随机抽样和分层抽样。随机抽样随机地采样数据,因此,只是根据上面计算出的样本大小随机选择25张发票即可。分层抽样是一种有考虑的系统的方法,试图降低采样数据的方差。例如,假设发票的总数是250000张,由于样本大小需要包括25发票,所以就可以将所有的发票分为25层,每层10000张,最后从每一层中随机抽取一个样本即可。
\end{shaded}
\vspace{-1em}

